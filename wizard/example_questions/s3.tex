In order for the resultant force to be a force with magnitude \SI{30}{\N} oriented along the horizontal, the following conditions must be true,
\begin{align}
	T_2 \cos\alpha + T_1 \cos\ang{20} &= \SI{30}{\N}	\\
	T_2 \sin\alpha - T_1 \sin\ang{20} &= 0 \,.
\end{align}
Hence,
\begin{align}
	T_2	&=	T_1\frac{\sin\ang{20}}{\sin\alpha}	\label{A1:eq:Q3:T1}\\
	\left(	T_1 \frac{\sin\ang{20}}{\sin\alpha}	\right) \cos\alpha &+ T_1\cos\ang{20} = \SI{30}{\N}	\nonumber \\
	T_1	&=	\frac{\SI{30}{\N}}{\frac{\sin\ang{20}}{\tan\alpha} + \cos\ang{20}}	\,. \label{A1:eq:Q3:T2}
\end{align}
When $\alpha=\ang{30}$, this yields
\begin{align*}
	T_1&=\SI{19.58}{\N}\\
	T_2&=\SI{13.39}{\N}
\end{align*}

Substituting the expression for $T_1$, equation~\ref{A1:eq:Q3:T1}, into the expression for $T_2$, equation~\ref{A1:eq:Q3:T2} allows us to determine $T_2$ as a function of $\alpha$,
\begin{align}
	T_2	&=\frac{\SI{30}{\N}}{\frac{\sin\ang{20}}{\tan\alpha} + \cos\ang{20}}	\frac{\sin\ang{20}}{\sin\alpha}	\nonumber	\\
		&=\frac{\SI{30}{\N}}	{\sin\ang{20}\frac{\cos\alpha}{\sin\alpha} + \cos\ang{20}}	\frac{\sin\ang{20}}{\sin\alpha}	\nonumber \\
		&=\frac{\SI{30}{\N}}	{\sin\ang{20}\frac{\cos\alpha}{\sin\alpha}\frac{\sin\alpha}{\sin\ang{20}} + \cos\ang{20}\frac{\sin\alpha}{\sin\ang{20}}}	\nonumber \\
		&=\frac{\SI{30}{\N}}	{\cancel{\sin\ang{20}}	\frac{\cos\alpha}	{\cancel{\sin\alpha}}	\frac{\cancel{\sin\alpha}}{\cancel{\sin\ang{20}}} + \cos\ang{20}\frac{\sin\alpha}{\sin\ang{20}}}	\nonumber \\
		&=\frac{\SI{30}{\N}}{\cos\alpha + \frac{\sin\alpha}{\tan\ang{20}}} \,.
\end{align}
Examining the form of this, we can see that $T_2$ will be minimal at $\alpha_{min}$ when $\left(\cos\alpha + \frac{\sin\alpha}{\tan\ang{20}}\right)$ is maximal. Searching for maximal stationary points in the usual manner,
\begin{align}
	\frac{\mathrm{d}}{\mathrm{d}\alpha} \left(\cos\alpha + \frac{\sin\alpha}{\tan\ang{20}}\right)	&=	0	\nonumber\\
		&=	-\sin\alpha_{min} +\frac{\cos\alpha_{min}}{\tan\ang{20}}	\nonumber	\\
		\tan\alpha_{min}	&=	\frac{1}{\tan\ang{20}}	\label{A1:eq:Q3:alpha}\\
		\Rightarrow\,\alpha_{min} &= \ang{70} \nonumber
\end{align}
Hence, $T_2$ is minimal when $T_2$ is perpendicular to $T_1$.\sidenote[][]{This is true for all angles of $T_2$, see if you can convince yourself of this by examining equation~\ref{A1:eq:Q3:alpha} analytically.}
\clearpage