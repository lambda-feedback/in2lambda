The angle $\alpha$ between the rope and the horizontal line can be calculated using the relation between $\tan\alpha$ and the lengths of the sides of the right angled triangle as shown in figure~\ref{A1:fig:Q1b},
\begin{align*}
	\tan\alpha	&=\frac{\SI{6}{\m}}{\SI{8}{\m}}\\
				&=0.75 \, ,	
\end{align*}
from which $\alpha=\ang{36.87}$.
\begin{marginfigure}
\centering
\includegraphics[width=\columnwidth]{problem1_1b.png}
\caption{The right angle triangle formed by the rope.}
\label{A1:fig:Q1b}
\end{marginfigure}

The horizontal component, $F_H$, is the magnitude of the force projected onto the horizontal axis,
\begin{align*}
	F_H	&=\SI{300}{\N} \cos\alpha \\
		&= \SI{240}{\N} \,,
\end{align*} and the vertical component, $F_V$, is the magnitude of the force projected onto the vertical axis, 
\begin{align*}
	F_V	&=\SI{300}{\N} \sin\alpha \\
		&= \SI{180}{\N}\, .
\end{align*}
\clearpage
