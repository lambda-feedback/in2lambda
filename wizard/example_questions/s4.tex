We can make many problems easier by choosing a suitable set of axes. Looking at the special directions of this system, we can choose to orient the x-axis parallel to the line connecting point $O$ and the point where the force acts and the y-axis perpendicular to this, as shown in figure~\ref{A1:fig:Q4b}. The moment of the force is then given by
\begin{align*}
	M	&=	F_y\, d	\\
		&=	\SI{30}{\N} \sin\ang{20} \SI{3}{\m}\\
		&=	\SI{30.8}{\N} \,.
\end{align*}
\begin{figure}
	\centering
	\includegraphics[width=0.8\columnwidth]{problem1_4b.png}
	\caption{The x-axis oriented parallel to the line between $O$ and the force's point of action.}
	\label{A1:fig:Q4b}
\end{figure}
We can think of this in two ways. From one viewpoint the moment is the component of the force parallel to the y-axis, multiplied by the distance from the pivot. From the other viewpoint, the moment is the whole force multiple by the perpendicular distance from the pivot. Note that these are equivalent, it just depends where you choose to group the $\sin\ang{20}$ term. This is because only the component of the force that is non-parallel to the line joining the point of action and the pivot will induce rotation.
\clearpage