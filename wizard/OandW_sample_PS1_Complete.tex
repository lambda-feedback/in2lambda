\documentclass[12pt]{article}

\input{../../Shared/header}

\pagestyle{empty}
\setlength{\textheight}{9.6in}  \setlength{\textwidth}{6.5in}
\setlength{\oddsidemargin}{-0.15in} \setlength{\evensidemargin}{0.2in}
\setlength{\topmargin}{-50pt}

\setlength{\arraycolsep}{2pt}
\setlength{\parskip}{0pt}
\setlength{\parindent}{0pt}

\begin{document}

\centerline{\Large Complex Numbers, Functions and Ordinary Differential Equations}
\bigskip\bigskip


\noindent
Problem Sheet 1\hfill 2019

\vskip-9pt\hrulefill


\bigskip




%\subsubsection*{Exercises}

\begin{enumerate}

\item Find the real and imaginary parts of:

$
\begin{array}[h!]{lll}
{\rm (a)}\hskip5pt 8+3\,i\hskip24pt&
{\rm (b)}\hskip5pt 4-15\,i\hskip24pt&
{\rm (c)}\hskip5pt \cos\theta-i\,\sin\theta\\
\noalign{\vskip12pt}
{\rm (d)}\hskip5pt i^2&
{\rm (e)}\hskip5pt i\,(2-5\,i)&
{\rm (f)}\hskip5pt (1+2\,i)(2-3\,i)
\end{array}
$

\item  Write each of the following expressions as a complex number in the form $x+i\,y$:

$
\begin{array}[h!]{lll}
{\rm (a)}\hskip5pt (5-i)(2+3\,i)\hskip24pt&
{\rm (b)}\hskip5pt (3-4\,i)(3+4\,i)\hskip24pt&
{\rm (c)}\hskip5pt (1+2\,i)^2\\
\noalign{\vskip12pt}
{\rm (d)}\hskip5pt \displaystyle{10\over4-2\,i}&
{\rm (e)}\hskip5pt \displaystyle{3-i\over4+3\,i}&
{\rm (f)}\hskip5pt \displaystyle{1\over i}
\end{array}
$


\item Define $z=(5+7\,i)(5+b\,i)$.

$
\begin{array}[h!]{l}
{\rm (a)}\hskip5pt {\rm If}\ b\ {\rm and}\ z\ {\rm are\ both\ real,\ find}\ b.\\
\noalign{\vskip12pt}
{\rm (b)}\hskip5pt {\rm If}\ {\rm Im}(b)={4\over5},\ {\rm and}\ z\ {\rm is\ pure\ imaginary,\ find}\
{\rm Re}(b).
\end{array}
$

\item Plot the following complex numbers in the complex plane:

$\begin{array}[h!]{l}
{\rm (a)}\hskip5pt 2\,i\hskip24pt
{\rm (b)}\hskip5pt -3+i\,2\hskip24pt
{\rm (c)}\hskip5pt (-3+i\,2)^\ast\hskip24pt
{\rm (d)}\hskip5pt \displaystyle{1+i\over\sqrt{2}}\hskip24pt
\end{array}
$

\item Write the following numbers in polar form:

$\begin{array}[h!]{l}
{\rm (a)}\hskip5pt i\hskip24pt
{\rm (b)}\hskip5pt -i\hskip24pt
{\rm (c)}\hskip5pt 1+i\hskip24pt
{\rm (d)}\hskip5pt 1-i\,\sqrt{3}\hskip24pt
\end{array}
$

\item Write the following complex numbers in Cartesian coordinates:

$\begin{array}[h!]{l}
{\rm (a)}\hskip5pt e^{-3\pi i/4}\hskip24pt
{\rm (b)}\hskip5pt e^{5\pi i/4}\hskip24pt
{\rm (c)}\hskip5pt 3\,e^{i}\hskip24pt
{\rm (d)}\hskip5pt \displaystyle{1\over\sqrt{3}\,e^{\pi i/3}}\hskip24pt
\end{array}
$

\item Consider the polar form of a complex number $z=r\,e^{i\,\theta}$.  Show that
$(z^2)^\ast=(z^\ast)^2$.

\item
\begin{enumerate}
\item Consider the polynomial $az^2+bz+c$, where $a$, $b$,
  and $c$ may be complex and $a\ne 0$. Given that $z_0$ and $z_0^\ast$ are distinct solutions to
$az^2 + bz + c = 0$ and thus are also solutions to $z^2+(b/a)z+(c/a)=z^2+b'z+c'=0$, show that $b'$ and $c'$ must be real.
  Conversely, if $a$, $b$ and $c$ are all real, and $z_0$ is a solution of $az^2 + bz + c = 0$, deduce that $z_0^\ast$ is also a solution.

\item Let $b=c=1$ and $a=t$, $t>\frac{1}{4}$. Make a graph of the
  pattern that the solutions to this equation traces out in the
  complex plane (hint: work out $|z+1|$).

\item Derive functions $a(t)$, $b(t)$ and $c(t)$ for which
  the solutions to the equation traces out a pattern which is rotated
  by $90^\circ$ counterclockwise around the origin with respect to the previous
  pattern (n.b. it is no longer the case that $a$, $b$ and $c$ are all real. Why?).
  How can you achieve an arbitrary rotation?
\end{enumerate}
\end{enumerate}


\end{document}
\documentclass[12pt]{article}

\input{../../Shared/header}

\pagestyle{empty}
\setlength{\textheight}{9.6in}  \setlength{\textwidth}{6.5in}
\setlength{\oddsidemargin}{-0.15in} \setlength{\evensidemargin}{0.2in}
\setlength{\topmargin}{-50pt}

\setlength{\arraycolsep}{2pt}
\setlength{\parskip}{0pt}
\setlength{\parindent}{0pt}

\begin{document}

\centerline{\Large Complex Numbers, Functions and Ordinary Differential Equations}
\bigskip\bigskip



\noindent
Solutions to Problem Sheet 1\hfill 2023

\vskip-9pt\hrulefill

%\subsubsection*{Homework}

\begin{enumerate}

\item For a complex number $a+i\,b$, in which $a$ and $b$ are real, the real and imaginary parts
are given by ${\rm Re}(a+i\,b)=a$ and ${\rm Im}(a+i\,b)=b$, respectively.  Thus,

$
\begin{array}[h!]{l}
{\rm (a)}\hskip5pt {\rm Re}(8+3\,i)=8\, ,\qquad{\rm Im}(8+3\,i)=3\, .\\
\noalign{\vskip12pt}
{\rm (b)}\hskip5pt {\rm Re}(4-15\,i)=4\, ,\qquad {\rm Im}(4-15\,i)=-15\, .\\
\noalign{\vskip12pt}
{\rm (c)}\hskip5pt {\rm Re}(\cos\theta-i\,\sin\theta)=\cos\theta\, ,\qquad {\rm
Im}(\cos\theta-i\,\sin\theta)=-\sin\theta\, .\\
\noalign{\vskip12pt}
{\rm (d)}\hskip5pt i^2=-1.\quad  {\rm Re}(i^2)=-1\, ,\qquad {\rm Im}(i^2)=0\, .\\
\noalign{\vskip12pt}
{\rm (e)}\hskip5pt i\,(2-5\,i)=5+2\,i.\qquad  {\rm Re}(5+2\,i)=5\, ,\qquad {\rm Im}(5+2\,i)=2\, .\\
\noalign{\vskip12pt}
{\rm (f)}\hskip5pt (1+2\,i)(2-3\,i)=2-3\,i+4\,i+6=8+i\, .\qquad {\rm Re}(8+i)=8\, ,\qquad {\rm
Im}(8+i)=1\, .
\end{array}
$

\item Applying the rules for the multiplication and division of
  complex numbers yields:

$
\begin{array}[h!]{lll}
{\rm (a)}\hskip5pt (5-i)(2+3\,i)=10+15\,i-2\,i+3=13+13\,i\, .\\
\noalign{\vskip12pt}
{\rm (b)}\hskip5pt (3-4\,i)(3+4\,i)=9+12\,i-12\,i+16=25\, .\\
\noalign{\vskip12pt}
{\rm (c)}\hskip5pt (1+2\,i)^2=(1+2\,i)(1+2\,i)=1+2\,i+2\,i-4=-3+4\,i\, .\\
\noalign{\vskip12pt}
{\rm (d)}\hskip5pt \displaystyle{{10\over4-2\,i}={10\over4-2\,i}\times{4+2\,i\over4+2\,i}=
{40+20\,i\over16+8\,i-8\,i+4}={40+20\,i\over20}=2+i\, .}\\
\noalign{\vskip12pt}
{\rm (e)}\hskip5pt \displaystyle{{3-i\over4+3\,i}={3-i\over4+3\,i}\times{4-3\,i\over4-3\,i}=
{12-9\,i-4\,i-3\over16-12\,i+12\,i+9}={9-13\,i\over25}={9\over25}-i\,{13\over25}\, .}\\
\noalign{\vskip12pt}
{\rm (f)}\hskip5pt \displaystyle{{1\over i}={1\over i}\times{-i\over-i}=-i\, .}
\end{array}
$

\item We have that $z=(5+7\,i)(5+b\,i)=25+5b\,i+35\,i-7b$.

$
\begin{array}[h!]{l}
{\rm (a)}\hskip5pt {\rm If}\ b\ {\rm and}\ z\ {\rm are\ both\ real},\ {\rm then\ the \ imaginary\
parts\ of\ both\ quantities\ vanish.}\ {\rm Thus,}\\
\noalign{\vskip12pt}
\hskip20pt {\rm Im}(z)=35+5b=0, {\rm so}\ b=-7.\\
\noalign{\vskip12pt}
{\rm (b)}\hskip5pt {\rm If}\ {\rm Im}(b)={4\over5},\ {\rm and}\ z\ {\rm is\ pure\ imaginary,}\
{\rm then\ the\ real\ part\ of}\ z\ {\rm vanishes\!:}\\
\noalign{\vskip12pt}
\hskip20pt{\rm Re}(z)=25+5\bigl[i\,{\rm Im}(b)\bigr]i-7\,{\rm Re}(b)=25-4-7\,{\rm Re}(b)=21-7\,{\rm
Re}(b)=0,\\
\noalign{\vskip12pt}
\hskip20pt{\rm so}\ {\rm Re}(b)=3.
\end{array}
$


\item The graphical representation of the complex number $z=x+i\,y$ is
  the point $(x,y)$ on a set of axes where the $x$-axis corresponds to
  the real part of the complex number and the $y$-axis the imaginary
  part. The required points are

  \begin{center}
    \includegraphics[width=8cm]{PS1-2}
  \end{center}

\item For a complex number $z=x+i\,y$, the polar form is $z=r\,e^{i\theta}$, where
\begin{equation*}
r=\sqrt{x^2+y^2}\, ,\qquad
\theta={\rm tan}^{-1}\biggl({y\over x}\biggr)\, .
\end{equation*}

$\begin{array}[h!]{l}
{\rm (a)}\hskip5pt z=i.\  {\rm We\ have\ that}\ x=0\ {\rm and}\ y=1,\ {\rm so}\ r=1,\ {\rm and}\
\theta={1\over2}\pi.\ {\rm Thus}\ i=e^{i\,\pi/2}.\\
\noalign{\vskip12pt}
{\rm (b)}\hskip5pt z=-i.\ {\rm We\ have\ that}\ x=0\ {\rm and}\ y=-1,\ {\rm so}\ r=1,\ {\rm and}\
\theta={3\over2}\pi.\ {\rm Thus}\ -i=e^{3i\,\pi/2}.\\
\noalign{\vskip12pt}
{\rm (c)}\hskip5pt z=1+i.\ {\rm We\ have\ that}\ x=1\ {\rm and}\ y=1,\ {\rm so}\ r=\sqrt{2},\ {\rm
and}\ \theta={1\over4}\pi.\\
\noalign{\vskip12pt}
\hskip20pt{\rm Thus}\ 1+i=\sqrt{2}\,e^{i\,\pi/4}.\\
\noalign{\vskip12pt}
{\rm (d)}\hskip5pt z=1-i\,\sqrt{3}.\ {\rm We\ have\ that}\ x=1\ {\rm and}\ y=-\sqrt{3},\ {\rm so}\
r=2,\ {\rm and}\\
\noalign{\vskip12pt}
\hskip20pt\theta={\rm tan}^{-1}(-\sqrt{3})=-{1\over3}\pi.\ {\rm Thus}\
1-i\,\sqrt{3}=2\,e^{-i\,\pi/3}.
\end{array}
$


\item A complex number $z=r\,e^{i\,\theta}$ can be written as
  $z=r\cos\theta+i\,r\sin\theta$.  Thus,


$\begin{array}[h!]{l}
{\rm (a)}\hskip5pt e^{-3\pi i/4}=\cos\bigl({3\over4}\pi\bigr)-i\,\sin\bigl({3\over4}\pi\bigr)=
\displaystyle{-{1+i\over\sqrt{2}}}=-{1\over2}\sqrt{2}-i\,{1\over2}\sqrt{2}\, .\\
\noalign{\vskip12pt}
{\rm (b)}\hskip5pt e^{5\pi i/4}=\cos\bigl({5\over4}\pi\bigr)+i\,\sin\bigl({5\over4}\pi\bigr)=
\displaystyle{-{1+i\over\sqrt{2}}}=-{1\over2}\sqrt{2}-i\,{1\over2}\sqrt{2}\, .\\
\noalign{\vskip12pt}
{\rm (c)}\hskip5pt 3\,e^{i}=3\cos 1+i\,\sin 1\, .\\
\noalign{\vskip12pt}
{\rm (d)}\hskip5pt \displaystyle{{1\over\sqrt{3}\,e^{\pi i/3}}=
{\sqrt{3}\over3}\,e^{-\pi i/3}=
{\sqrt{3}\over3}\cos\bigl({\textstyle{1\over3}}\pi\bigr)-
{i\,\sqrt{3}\over3}\sin\bigl({\textstyle{1\over3}}\pi\bigr)=
{\sqrt{3}\over6}-{i\over2}}\, .
\end{array}
$

%\end{enumerate}

%\subsubsection*{Tutorial}


\item Given that $z=r\,e^{i\,\theta}$, we have
\begin{eqnarray*}
(z^2)^\ast=\bigl[\bigl(r\,e^{i\,\theta}\bigr)^2\bigr]^\ast=\bigl(r^2\,e^{2i\,\theta}\bigr)^\ast=r^2\,e^{-2i\,\theta}=\bigl(r\,e^{-i\,\theta}\bigr)^2=
(z^\ast)^2\, .
\end{eqnarray*}

\item
\begin{enumerate}
\item Suppose $z_0$ is a solution of $az^2+bz+c=0$, and therefore also of $z^2+b'z+c'=0$ where $b'=b/a$ and $c'=c/a$ and $a\ne0$. Given that  $z_0^\ast$ is also a solution implies that
\[
(z-z_0)(z-z_0^\ast)=0~,
\]
or
\[
z^2-(z_0+z_0^\ast)z+|z_0|^2=0~,
\]
showing that $b'$ and $c'$ must be real. Conversely, if $a$, $b$, and $c$ are real, and $z_0$ is a solution of $az^2+bz+c=0$, then $az_0^2+bz_0+c=0$, and taking the conjugate yields
\[
a(z_0^2)^\ast+bz_0^\ast+c=0~,
\]
or
\[
a(z_0^\ast)^2+bz_0^\ast+c=0~,
\]
showing that $z_0^\ast$ is also a solution.

\item The roots of the polynomial are
  \[
  z_{\pm} = \frac{1}{2t} \left( -1 \pm \sqrt{1 - 4t} \right)
  \]
  which for $t>\frac{1}{4}$ has complex roots,
  \[
  z_{\pm} = - \frac{1}{2t} \pm i\, \frac{1}{2t}\sqrt{4t - 1} \, .
  \]
  The solutions will map out a circle centered at $z=-1$ in the complex plane.
  (To see this either show that $x={\rm Re}(z), y={\rm Im}(z)$ satisfies $(x+1)^2+y^2=1$,
  or that $|z+1|^2=1$.)
  \begin{center}
    \includegraphics[width=8cm]{PS1-1}
  \end{center}

\item Rotating the figure through $90^\circ$ is equivalent to multiplying both
  solutions by $e^{i\pi/2} = i$, thus
  \begin{align}
    z_{\pm} & = i \left( - \frac{1}{2t} \pm i\, \frac{1}{2t}\sqrt{4t - 1} \right)
    \nonumber \\
      & = \pm \frac{1}{2t}\sqrt{4t - 1} - \frac{i}{2t} \, . \nonumber
  \end{align}
  Forming the polynomial by multiplying together the two solutions gives
  \begin{align}
    p & =\left(z-z_+\right)\left(z-z_-\right)\nonumber\\
      &=
    \left( z - \frac{1}{2t}\sqrt{4t - 1} + \frac{i}{2t} \right)
    \left( z + \frac{1}{2t}\sqrt{4t - 1} + \frac{i}{2t} \right)
    \nonumber \\
     & = -\frac{1}{t}(-t z^2 - i\,z  + 1) \nonumber \, ,
  \end{align}
  thus reducing $p=0$ to $-t z^2 - i\,z + 1=0$.
  %It can be seen that this is just the original polynomial with $z$ substituted with $iz=e^{i\pi/2}z$.
  So one solution is $a(t)=-t$, $b=-i$ and $c=1$. All other solutions are multiples of this set. The solutions that were conjugate pairs for the original equation are not conjugate pairs for the new equation as the coefficients of the latter are no longer real.
  An arbitrary rotation of angle $\theta$ can be made by substituting $z_{\pm}$ with $e^{i\theta}z_{\pm}$.
\end{enumerate}
\end{enumerate}



\end{document}
