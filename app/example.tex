\documentclass[12pt]{article}


\pagestyle{empty}
\setlength{\textheight}{9.6in}  \setlength{\textwidth}{6.5in}
\setlength{\oddsidemargin}{-0.15in} \setlength{\evensidemargin}{0.2in}
\setlength{\topmargin}{-50pt}

\setlength{\arraycolsep}{2pt}
\setlength{\parskip}{0pt}
\setlength{\parindent}{0pt}

\begin{document}

\centerline{\Large Complex Numbers, Functions and Ordinary Differential Equations}
\bigskip\bigskip


\noindent
Problem Sheet 1\hfill 2019

\vskip-9pt\hrulefill


\bigskip




%\subsubsection*{Exercises}

\begin{enumerate}

\item Find the real and imaginary parts of:

$
\begin{array}[h!]{lll}
{\rm (a)}\hskip5pt 8+3\,i\hskip24pt&
{\rm (b)}\hskip5pt 4-15\,i\hskip24pt&
{\rm (c)}\hskip5pt \cos\theta-i\,\sin\theta\\
\noalign{\vskip12pt}
{\rm (d)}\hskip5pt i^2&
{\rm (e)}\hskip5pt i\,(2-5\,i)&
{\rm (f)}\hskip5pt (1+2\,i)(2-3\,i)
\end{array}
$

\item  Write each of the following expressions as a complex number in the form $x+i\,y$:

$
\begin{array}[h!]{lll}
{\rm (a)}\hskip5pt (5-i)(2+3\,i)\hskip24pt&
{\rm (b)}\hskip5pt (3-4\,i)(3+4\,i)\hskip24pt&
{\rm (c)}\hskip5pt (1+2\,i)^2\\
\noalign{\vskip12pt}
{\rm (d)}\hskip5pt \displaystyle{10\over4-2\,i}&
{\rm (e)}\hskip5pt \displaystyle{3-i\over4+3\,i}&
{\rm (f)}\hskip5pt \displaystyle{1\over i}
\end{array}
$


\item Define $z=(5+7\,i)(5+b\,i)$.

$
\begin{array}[h!]{l}
{\rm (a)}\hskip5pt {\rm If}\ b\ {\rm and}\ z\ {\rm are\ both\ real,\ find}\ b.\\
\noalign{\vskip12pt}
{\rm (b)}\hskip5pt {\rm If}\ {\rm Im}(b)={4\over5},\ {\rm and}\ z\ {\rm is\ pure\ imaginary,\ find}\
{\rm Re}(b).
\end{array}
$

\item Plot the following complex numbers in the complex plane:

$\begin{array}[h!]{l}
{\rm (a)}\hskip5pt 2\,i\hskip24pt
{\rm (b)}\hskip5pt -3+i\,2\hskip24pt
{\rm (c)}\hskip5pt (-3+i\,2)^\ast\hskip24pt
{\rm (d)}\hskip5pt \displaystyle{1+i\over\sqrt{2}}\hskip24pt
\end{array}
$

\item Write the following numbers in polar form:

$\begin{array}[h!]{l}
{\rm (a)}\hskip5pt i\hskip24pt
{\rm (b)}\hskip5pt -i\hskip24pt
{\rm (c)}\hskip5pt 1+i\hskip24pt
{\rm (d)}\hskip5pt 1-i\,\sqrt{3}\hskip24pt
\end{array}
$

\item Write the following complex numbers in Cartesian coordinates:

$\begin{array}[h!]{l}
{\rm (a)}\hskip5pt e^{-3\pi i/4}\hskip24pt
{\rm (b)}\hskip5pt e^{5\pi i/4}\hskip24pt
{\rm (c)}\hskip5pt 3\,e^{i}\hskip24pt
{\rm (d)}\hskip5pt \displaystyle{1\over\sqrt{3}\,e^{\pi i/3}}\hskip24pt
\end{array}
$

\item Consider the polar form of a complex number $z=r\,e^{i\,\theta}$.  Show that
$(z^2)^\ast=(z^\ast)^2$.

\item
\begin{enumerate}
\item Consider the polynomial $az^2+bz+c$, where $a$, $b$,
  and $c$ may be complex and $a\ne 0$. Given that $z_0$ and $z_0^\ast$ are distinct solutions to
$az^2 + bz + c = 0$ and thus are also solutions to $z^2+(b/a)z+(c/a)=z^2+b'z+c'=0$, show that $b'$ and $c'$ must be real.
  Conversely, if $a$, $b$ and $c$ are all real, and $z_0$ is a solution of $az^2 + bz + c = 0$, deduce that $z_0^\ast$ is also a solution.

\item Let $b=c=1$ and $a=t$, $t>\frac{1}{4}$. Make a graph of the
  pattern that the solutions to this equation traces out in the
  complex plane (hint: work out $|z+1|$).

\item Derive functions $a(t)$, $b(t)$ and $c(t)$ for which
  the solutions to the equation traces out a pattern which is rotated
  by $90^\circ$ counterclockwise around the origin with respect to the previous
  pattern (n.b. it is no longer the case that $a$, $b$ and $c$ are all real. Why?).
  How can you achieve an arbitrary rotation?
\end{enumerate}
\end{enumerate}


\end{document}
